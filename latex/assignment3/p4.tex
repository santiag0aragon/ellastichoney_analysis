ISPs:

The costs involved in cleaning up of botnet infected systems affect ISP’s behaviour. If the cost is sufficiently low, ISP can have an incentive to voluntarily help its users without disconnecting those users. If the cost is not low enough for ISP to have an incentive to do this, imposing liability on ISP can have the following two effects on consumer surplus. The first effect is an increase in consumer surplus. Imposing liability on ISP removes negative externalities from the network and makes accessing the network more attractive, which results in an increase in on-line users and thus positive externalities. The second one is a decrease in consumer surplus. Imposing liability on ISP raises access fee or purges botnet-infected users, which result in a decrease in on-line users and thus positive externalities. If the clean-up cost is sufficiently high, the latter effect dominates the former one, and thus imposing liability on ISP results in decrease of both ISP’s profits and consumer surplus. If ISPs can have an incentive to disconnect users vulnerable to botnet malware even without liability if users’ preferences for precautions against malware is sufficiently different. In this case, securing the network by disconnecting vulnerable users makes it possible to charge other users the access fee high enough to increase ISP’s profits. However, if the clean-up cost is sufficiently low, cleaning up malware from infected computers without disconnecting any users can be more profitable to ISP because letting vulnerable users be on-line makes positive externalities larger than disconnecting them~\cite{kinukawa2012should}.

Cloud Providers:



End Users:
Home and SMB users create the most negative externalities which is partly absorbed by ISPs (Internet service providers) and FSPs (financial service providers). The risky online behaviour in combination with not employing security software would result in this externality. This can happen  if an user doesn’t understand how malware works, and often don’t know when they become infected. They do not understand the degree of harm botnet infection can potentially cause and they don't consider paying security software as economical~\cite{asghari2010botnet}. 
