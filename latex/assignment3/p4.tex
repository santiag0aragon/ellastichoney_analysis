\begin{itemize}
\item \textit{Vendors of Operating Systems and Softwares. } Vendors of Operating Systems and software create positive externalities since their role to mitigate the security issue can have a big impact to lower the cost of security investments by ISPs and other actors. In most cases software monopolies limit the available product choice and thus it becomes a moral and social responsibility of organizations to take appropriate measure to ensure security considerations in their software products. Vendors look at what insecure software costs them instead of the total cost of insecure software because, they miss a lot of the costs: all the money we, the software product buyers, are spending on security.
\item \textit{End Users.} Botnet infected machines create negative externalities caused by risky online behavior and not deploying preventive controls such as antivirus and security software. These externalities are partly absorbed by ISPs, organizations and other users. A botnet herder who may infect thousands of other users may end up playing a key part of the harm being felt by other users. The harm of those machines is not just restricted to the infected computers, but in-fact often used for other purposes to send spam, to infect other computers and to launch denial of service attacks. Therefore, there is a very less incentive for the botnet-infected user to clean up because they do not actually experience much harm themselves.




\item \textit{ISPs.} ISPs tend to have an incentive to voluntarily help its users in cleaning up of botnet infected systems if the cost involved is sufficiently low. Otherwise, imposing liability on ISPs can have the following two effects on consumer surplus. The first effect would remove negative externalities from the network and makes accessing the network more attractive, which results in an increase in on-line users and thus positive externalities. This would cause an increase in consumer surplus. The second effect would raise access fee or purges botnet-infected users, which result in a decrease in on-line users and thus positive externalities. This would cause a decrease in consumer surplus. Also, if the clean-up cost is sufficiently low, cleaning up malware from infected computers without disconnecting any users can be more profitable to ISPs because letting vulnerable users be on-line makes positive externalities larger than disconnecting them. However, If the clean-up cost is sufficiently high, securing the network by disconnecting vulnerable users makes it possible to increase their profits by charging other users a high access fee~\cite{kinukawa2012should}.

% Cloud Providers:



% End Users:
% Home and SMB users create the most negative externalities which is partly absorbed by ISPs (Internet service providers) and FSPs (financial service providers). The risky online behaviour in combination with not employing security software would result in this externality. This can happen  if an user doesn’t understand how malware works, and often don’t know when they become infected. They do not understand the degree of harm botnet infection can potentially cause and they don't consider paying security software as economical~\cite{asghari2010botnet}.
\end{itemize}
