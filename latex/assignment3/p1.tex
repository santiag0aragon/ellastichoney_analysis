%!TEX root = main.tex
\label{cap:sec1}
\begin{itemize}
\item \textit{Vendors of Operating Systems and Softwares.} Vendors of Operating Systems should develop and ship free specialist applications along with their products to prevent, detect and mitigate the impact of malware that are used for botnet infections. These applications should be able to detect malicious code and traffic in real-time and update the root server for newly found threats to help the general users to get updated from one newly found threat.
\item \textit{End Users.} End users can attend free online awareness campaigns and trainings to have better awareness of the impacts of botnet infections and change their operating and work habits while browsing and downloading online content. Users should be aware to keep their operating systems and application software updated. Their conscious approach for being vulnerable to security threats may help the users to browse the Internet in a safe and secure manner, helping to mitigate the growth of botnet infections.



% Botnet infection is a serious threat for a number of entities: end users, businesses, websites, Internet Service Providers (ISPs), Cloud Service Providers and Law Enforcement Agencies. Consequently, thwarting botnets would benefit all these entities. However, the problem of botnet infection in general cannot be solved exclusively by any of these entities alone. In this direction, we propose that each entity should take measures to fight the botnet growth.

\item \textit{Internet Service Providers (ISPs).}  ISPs can detect the systems infected with malicious bots, notify respective users and use various remediation techniques like removing, disabling, or otherwise rendering a bot harmless to address the problem. Since ISPs provide IP connectivity, they can act upon the bots' traffic and can control bots' access to end users. ISPs can follow sets of best practices for the remediation of bots prepared by Internet Engineering Taskforce (IETF) and the Messaging Anti-Abuse Working Group (MAAWG). ISPs can go for security policy changes at the network level, can use walled gardens to quarantine infected users, can filter Inbound and outbound email and  can distribute secure ICT infrastructure to users to mitigate botnet infection~\cite{charney2012collective}. Remediating the installations of malicious bots and mitigating the effects would make it more difficult for botnets to operate and would reduce the level of botnet activities on a particular Internet Service Provider's network~\cite{anderson2013measuring}. 


% 2)ISPs are in the best position to detect the presence of a botnet and to take measures against it. ISPS can use technical means that can slow the botnet down. An example for this would be consuming its resources. ISPs can take these measures by performing Denial of Service attacks against Command-and Control Servers of the botnet, trapping and holding connections from infected machines, or blocking of malicious domains~\cite{leder2009proactive}.

% \subsection{Cloud Providers}

% To have cloud infrastructure in which configuration constantly evolves to confuse attackers without significantly degrading the quality of service. Proposed solutions may increase the cost for potential attackers by complicating the attack process and limiting the exposure of network vulnerability in order to make the network more resilient against novel and persistent attacks. This can be done by using following technologies: Polymorphism- Develop the novel cloud defence polymorphism techniques to protect cloud infrastructures from attackers[Use botnet polymorphism techniques, i.e., server-side polymorphism and malware polymorphism], Agility- Develop the rapid provisioning technologies of cloud resources to provide high resource availability to cloud customers[Investigate botnet agility behaviours] and Poisoning Prevention- Develop the tamper evident technologies that make unauthorized access to the protected cloud resources easily detected[Probe botnet poisoning mechanisms]~\cite{peng2014moving}.

% \subsection{End Users}

% End-users comprise of individuals users and small to medium sized businesses. Home users have to start using antivirus and firewall software as part of personal botnet mitigation measures. Small to medium businesses should fight botnet in association with ISPs by coming up with various ‘best practice recommendations’ regarding Internet security. It is win-win for both the entities~\cite{asghari2010botnet}.
\end{itemize}
